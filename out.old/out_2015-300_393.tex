
\documentclass{templates}

\makeatletter
\def\@maketitle{%
  \newpage
  \null
 % \vskip 0.2em%
  \begin{center}%
  \let \footnote \thanks
	{ %\Huge \@title
 		\includegraphics[width=1\linewidth,height=0.08\linewidth]{Bandeau-lnsm} \\
 		\raggedleft \footnotesize Telephone labo:+261 (0) 20 24 513 88 \\ Email labo: lnsm@neuromg.org \\
    }%
   \end{center}%
   {
   \begin{flushright}
	 \footnotesize \textbf{Médecin examinateur:}Dr Hasina RAZAKAHARIMANANA	\\ \vskip 0.5em 
	 \textbf{Médecin demandeur:}Dr INZAKI AHMED\\
	 \textbf{Adresse professionnelle du demandeur:}Pediatrie CHUJRB\\
   \end{flushright}
   }
	\begin{flushleft}
	{ \vskip -7em \par \footnotesize
		  \textbf{Nom et prénoms:} 	RANDRIAMBELOTIANA Tendroavo Finoana  \\
		   \textbf{Sexe:} Masculin				\\
		   \textbf{Date de naissance:} 04/11/2009		\\
		   \textbf{Domicile:} Ampitatafika	\\
		   \textbf{Numéro ID de l'examen et date:} 2015-300-26/01/2015			\\
		 \vskip 1.6em
	  \hrule height 2pt\hfill}\par
	\end{flushleft}
}
\makeatother

%\onehalfspacing


%\title{Titre de l'article}

\begin{document}
\maketitle
\begin{Clinique}
\begin{center}

{\small \vskip -2.8em \textbf{Bilan neurologique electro-clinique}}
\end{center}

\section*{\vskip -2em Motif}
Controle devant crises convulsives
\section*{Bilan neurologique}
\subsection*{Histoire de la maladie}
\begin{itemize}
%\item[\textbf{Numéro ID du dernier examen EEG:}] 
%\item[\textbf{Date du dernier examen EEG:}] 
%\item[	extbf{Résultats du dernier examen EEG:}] 
\item [\textbf{Première crise:}]Première crise:3ans,type de crise:crises phonatoire  avec machonnement  sans généralisation secondaire , nombre :1,  circonstance :apyretique 
Pas de perte de contact  ni sursauts  nocturne ni diurne
%\item[\textbf{Traitement prescrit lors de la dernière visite:}] 
\item [\textbf{Evolution ultérieure:}]	
\item[\textbf{Dernière crise:}]-
\item[\textbf{Traitement actuel:}]DEPAKINE sb:100mg-100mg-200mg depuis 21 mois  (04.2013)
\end{itemize}
\subsection*{Antecedents}
\begin{itemize}
\item[\textbf{Grossesse, accouchement, trauma et ictère néonatal:}]
\item[\textbf{Développement staturo-pondéral:}]normal
\item[\textbf{Développement psychomoteur, social et scolaire:}]
\item[\textbf{Antécédents personnels:}]pas de crises convulsives fébrile de NRS ni  traumatisme cranien ni parasitose cérébrale ni méningite
\item[\textbf{Antécédents familiaux:}]
\item[\textbf{Traitements antérieurs:}]Aucun
\end{itemize}
\subsection*{Examen physique}
\begin{itemize}
\item[\textbf{Plaintes et comportement:}]persistance de l'hyperactivité psychomotrice
\item[\textbf{Etat général:}]Poids: 14kg
\item[\textbf{Etat neurologique:}]Motricité, sensibilité, vigilance-attention-fonctions  cognitives, nerfs cr\^aniens et sphincters lors de l'examen initial et  donner les résultats du jour 
\item[\textbf{Reste de l'examen:}] Rappeler  les résultats de l'examen initial et décrire l'es résultats de ce  jour 
Vidéo d'une crise 
\item[\textbf{Bilan biologique et bactériologique:}]Sang, urines, LCR, Selles,... de l'examen  initial puis de l'examen de ce jour. 
\end{itemize}
 
\subsection*{Examens complémentaires}
\begin{itemize}
\item[\textbf{Explorations fonctionnelles:}] EMG, EFR,..de l'examen initial puis de l'examen de ce jour.
\item[\textbf{Bilan morphologique:}] TDM, IRM, échographie, RX cr\^ane et  RxCP,...de l'examen initial puis de l'examen de ce jour. 
\item[\textbf{Bilan histologique et anatomo-pathologique:}]-
\end{itemize}
\subsection*{Synthèse clinique}
Enfant de 5 ans ,sans antécédent familial d'epilespie  ,est revenu pour controle EEG éffectué pour des crises partielles  complexes sans généralisation secondaire ,apyretique,inaugurales  ,évoluant de l'age de 3ans,non accompagnées de retard de  developpement psychomoteur ni difficulté scolaire,d'examen  neurologique normal , persistance de l'hyperactivité psychomotrice à  lm'examen clinique,bilan EEG montrant des tracés sensiblement  normaux, traitées par DEPAKINE: 28.57mg/kg/J en 3 prises depuis 21  mois .

\section*{Examen électro-encéphalographique}
\begin{itemize}
\item[\textbf{Technique:}]Appareil numérique, vitesse à $15\ mm/s$ et amplitude $70\ \mu V/cm$. Montage enfant/adulte, 16 voies, bipolaire, dérivations  longitudinales et transversales
\item[\textbf{Conditions d'examen:}]Eveillé, yeux ouverts, agité
\item[\textbf{Tests de réactivité:}]Enfant de 5 ans ,sans antécédent familial d'epilespie  ,est revenu pour controle EEG éffectué pour des crises partielles  complexes sans généralisation secondaire ,apyretique,inaugurales  ,évoluant de l'age de 3ans,non accompagnées de retard de  developpement psychomoteur ni difficulté scolaire,d'examen  neurologique normal , persistance de l'hyperactivité psychomotrice à  lm'examen clinique,bilan EEG montrant des tracés sensiblement  normaux, traitées par DEPAKINE: 28.57mg/kg/J en 3 prises depuis 21  mois .
\item[\textbf{Méthodes d'activation:}]Stimulation lumineuse intermittente
\item[\textbf{Artefacts:}] 
\item[\textbf{Activité de base:}] Décrire dans les  régions temporo-occipitales et pariéto-occipitales la présence  d'ondes alpha, sa fréquence, son amplitude, sa périodicité et  rythmicité, sa symétrie, sa synchronie et sa réactivité à  l'ouverture et à la fermeture des yeux ou de la main 
\item[\textbf{Activité au repos:}] Décrire la présence éventuelle de pointes ou  d'ondes lentes et en préciser le siège, la rythmicité, la symétrie  et la synchronie 
\item[\textbf{Activité après activation:}] Décrire l'apparition éventuelles d'une modification de  l'activité de base ou de pointes, voire d'ondes lentes, au cours de  l'épreuve de l'hyperpnée provoquée et la stimulation lumineuse  intermittente 
\item[\textbf{Autres anomalies rythmiques ou paroxystiques physiologiques:}]
\item[\textbf{Interprétation du tracé:}] Comparaison par  rapport au dernier tracé 
\end{itemize}
\section*{ Conclusion et recommandations}
\begin{itemize}
\item [\textbf{Conclusion:}] 
\item[\textbf{Recommandations:}] -
\end{itemize}
\end{Clinique}
\closing
\end{document}

