\documentclass{templates}

\makeatletter
\def\@maketitle{%
  \newpage
  \null
 % \vskip 0.2em%
  \begin{center}%
  \let \footnote \thanks
	{ %\Huge \@title
 		\includegraphics[width=1\linewidth,height=0.08\linewidth]{Bandeau_neuropsybefela} \\
 		\raggedleft \small Telephone labo:+261 (0) 20 24 513 88 \\ Email labo: lnsm@neuromg.org \\
    }%
   \end{center}%
   {
   \begin{flushright}
	 \small \textbf{Destinataire:} Dr BAKOSAHOLY
   \end{flushright}
   }
    {\raggedleft \small \@date \par \vskip 0.3em  \par}
	  
	\begin{flushleft}
	{ \vskip -7.5em \par \small 
		  \textbf{Nom et prénoms:} RAKOTONDRAPARANY Jean Ely	    \\
		   \textbf{Sexe:} Masculin 									\\
		   \textbf{Date de naissance:} 2002-07-20  	     			\\
		   \textbf{Domicile:} Soavinandrina Itasy   				\\
		   \textbf{Numéro ID de l'examen:} 2014-4637  				\\
		   \textbf{Date de l'examen:} 2014-12-03  			 \vskip 0.3em
	  \hrule height 2pt\hfill}\par
	\end{flushleft}
}
\makeatother


\newcommand{\closing}
{\raggedleft Je vous remercie pour votre confiance, \\ \vskip 3em \raggedleft Prof Alain D Tehindrazanarivelo}

\begin{document}
\maketitle
\begin{Clinique}
\begin{center}
\vskip -2em
{\Large \textbf{Compte-rendu d'hospitalisation}}
\end{center}
\section*{\vskip -2em Motif d'hospitalisation}
Trouble statique.  
\section*{Démarche diagnostique}
\subsection*{Histoire de la maladie}
Première crise: Trouble statique avec hémiparésie du pied gauche détecté par les parents, évoluant de l'âge de 11 ans.  
Notion de perte de contact, pas de sursauts nocturnes ni diurnes.\\
Évolution ultérieure: pas de récidive.\\
Traitement actuel: aucun. 
\subsection*{Antécédents}
Développement psychique et physique: normal, difficulté scolaire\\
Antécédents personnels: pas de convulsion fébrile de NRS ni traumatisme cranien ni parasitose cérébrale ni méningite.\\ 
Antécédents familiaux: pas d'antécédent familial d'épilepsie.
\subsection*{Examen physique}
Plaintes et comportement: pas de plainte particulière\\
Etat général: Poids:27.2kg\\
Etat neurologique: parésie du pied gauche, cotation :4/5\\
sensibilité : normale\\
pas de trouble de la vigilance ni atteintes des paires craniens ni troubles sphinctériens\\ 
Reste de l'examen: strabisme divergent de l'oeil droit
\subsection*{Examens complementaires}
Explorations fonctionnelles: EEG du 13.05.2014: anomalies lentes rolandiques et temporales des 2 cotés\\ 
Scanner axial sans contraste: non fait;
Scanner coronal sans contraste: non fait;
Scanner sagittal sans contraste:non fait;
\subsection*{Hypothèses diagnostiques}
Adolescent de 12 ans, sans antécédent familial d'epilepsie, presente un trouble statique avec hémiparésie du pied gauche, des crises d'absence, évoluant de l'age de 11 ans,non accompagnées de retard de developpement psychomoteur mais de difficulté scolaire, calme à l'examen clinique, pas de traitement actuel.
\section*{Bilan de confirmation diagnostique}
\subsection*{Critères diagnostiques}
Alpha.
\subsection*{Biologie}
Yeux fermés, non endormi.
\subsection*{Bilan morphologique}
Hyperpnée provoquée, Stimulation Lumineuse Intermittente.
\subsection*{Explorations fonctionnelles}
Hyperpnée provoquée, Stimulation Lumineuse Intermittente.
\subsection*{Diagnostic retenu}
Alpha.

\section*{Traitement reçu}
\subsection*{Base de recommandations}
Aucun grapho-element franchement pathologique.
\subsection*{Prescription}
Aucun grapho-element franchement pathologique.
\subsection*{Evolution}
Ralentissement global
\subsection*{Devenir}
Traces actuels normaux.

\section*{Conclusion et recommandations}
\subsection*{Conclusion}
Epilepsie idiopathique
\subsection*{Recommandations}
Depakine 20 mg/kg/j en 2 prises. Contrôle semestriel. Hygiène de vie.
\end{Clinique}
\closing
\end{document}

